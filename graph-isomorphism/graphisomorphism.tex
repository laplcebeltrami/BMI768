\documentclass[runningheads,a4paper]{llncs}

\usepackage{amsfonts,amssymb,amsmath}
%\setcounter{tocdepth}{3}
\usepackage{natbib}

\usepackage{wrapfig}
\usepackage{llncsdoc}
\usepackage{epsfig,psfrag}
\usepackage{latexsym}
\usepackage{longtable}
\usepackage{tikz,pgf}
\usepackage{subfig}
\usepackage{wrapfig}
%\usepackage{rotating}
%\usepackage{algorithm}
%\usepackage{algorithmic}

\usepackage{graphicx}
\usepackage{epstopdf}
\usepackage{color}

\newcommand{\bqn}{\begin{eqnarray}}
\newcommand{\eqn}{\end{eqnarray}}
\newcommand{\bq}{\begin{eqnarray*}}
\newcommand{\eq}{\end{eqnarray*}}
\usepackage[ruled,vlined]{algorithm2e}

\usepackage{url}
\usepackage{hyperref}
\hypersetup{colorlinks,%
citecolor=black,%
filecolor=blue,%
linkcolor=red,%
urlcolor=blue,%
pdftex}



\newcommand{\blue}[1]{{\color{blue} #1}}
\newcommand{\green}[1]{{\color{green} #1}}
\newcommand{\red}[1]{{\color{red} #1}}



\begin{document}



\title{P1: Statistical Inference on Graph Isomorphism}
\author{Moo K. Chung
%\hspace{0.2cm}$
%\footnote[0]{Corresponding address: Moo K. Chung, Waisman Center $\#$281, 1500 Highland Ave. Madison, WI.  53705. USA. Email:{\tt mkhung@wisc.edu}. \url{http://www.stat.wisc.edu/~mchung}}, 
 }
\institute{
University of Wisconsin-Madison, USA\\
\vspace{0.3cm}
\blue{\tt mkchung@wisc.edu}
}
\authorrunning{Chung}



\maketitle


%\pagestyle{empty}

%\pagestyle{headings}
%\setcounter{page}{1}
\pagenumbering{arabic}


\begin{abstract}
Two graphs are said to be isomorphic if there exists a bijection between their vertex sets that preserves adjacency. Graph isomorphism plays a fundamental role in graph matching, alignment, and representation learning, and is widely used in machine learning to compare network-structured data. However, exact isomorphism is rarely observed in real-world networks, particularly in biological and social systems, where noise, heterogeneity, and measurement variability break strict structural equivalence.

Rather than treating non-isomorphism as a failure of graph matching, we view it as an informative signal. In many applications, networks arise from populations that share a common underlying generative mechanism but differ through noise, perturbations, or disease-related effects. Within a homogeneous group, such as healthy controls, networks are therefore expected to remain closer to a common isomorphism class, whereas between heterogeneous groups, such as clinical populations, systematic deviations from this class may emerge. This perspective motivates quantifying the \emph{degree of deviation from isomorphism} as a statistical descriptor of population-level network organization.


In this work, we propose a principled framework for measuring divergence from graph isomorphism and demonstrate how this measure can be used to statistically characterize within-group consistency and between-group differences in network data. By reframing isomorphism from a binary property into a continuous, quantifiable notion, our approach enables population-level inference on network structure without requiring exact graph matching, providing a robust tool for statistical analysis of complex networks.
\end{abstract}




 
\bibliographystyle{agsm} 
\bibliography{reference.2026.01.26}

\end{document}
