\documentclass[12pt,portrait]{article}
\usepackage[round]{natbib}
\usepackage{times,amsmath,amsfonts}
\usepackage[dvips]{graphicx,graphics}


\usepackage{multirow}
\usepackage{amsfonts,amsmath,amssymb}
\usepackage{url}
\usepackage{hyperref}
\hypersetup{colorlinks,%
citecolor=black,%
filecolor=blue,%
linkcolor=red,%
urlcolor=black,%
pdftex}

%\newcommand{\blue}[1]{{\color{blue} #1}}
%\newcommand{\green}[1]{{\color{green} #1}}
%\newcommand{\red}[1]{{\color{red} #1}}

\newtheorem{theorem}{Theorem}
\newtheorem{definition}{Definition}
\newtheorem{example}{Example}

\newcommand{\bq}{\begin{eqnarray*}}
\newcommand{\eq}{\end{eqnarray*}}
\newcommand{\bqn}{\begin{eqnarray}}
\newcommand{\eqn}{\end{eqnarray}}

\usepackage{wrapfig}
\usepackage{epsfig,psfrag}

\begin{document}
%\pagenumbering{arabic}
\pagenumbering{gobble}

\title{How to Write Reports
}
\author{Moo K. Chung\\
University of Wisconsin-Madison\\
\quad \tt{mkchung@wisc.edu}}
\maketitle 

\begin{abstract}
This short article explains how to write the Final Project Report that is more than 10 pages excluding figures, tables and references. Including all, it should be about 15 pages or more (based on this template). This LeTex template can be used to write the report.  If you are not proficient in using LeTex, there is no need to use latex and MS-WORD or any other paper writing tool can be used. The abstract summarizes your research and its outcomes without technical details. 
\end{abstract}

\section{Introduction}


The introduction of the paper expands the abstract and list main contributions of the paper with respect to existing literature. This is where you do literature review explaining what similar works have been done by others. With that list of literature, you contrast your main contribution of the research. It is expected you list reasonable number of previous papers. Try to add at least 10-20 references and explain what is new or different about your method with respect to existing methods in literature. You need detailed review discussing cons and pros of existing methods against what you are trying to do. Also you need to able to spell out at least 2-3 major contributions or key points of your research.



\section{Materials and Methods}

\subsection{Data} The section explains in detail what the data set is. Enough detailed explanation about data and preprocessing on the data is a must. For instance, you need to explain about what scanner was used and what is the image resolution and image acquisition parameters are. If you have your own data, you can use it after consultation with the instructor.\\


\subsection{Methods}  Then you describe in great detail what methods you used in analyzing the data.  It is expected you will be using methods we studied in the class. \\

\subsection{Validation} This is where you justify your method using validation and comparisons against existing baseline methods using the real data and simulated data. Validate your methods with the ground truth data. Sufficient enough details should be added for any reader to duplicate the proposed method. Compare your methods to existing baseline methods. Demonstrate why your method is better than other methods. Also need to discuss how you will compare your method to existing methods. However, if your method is the first method to do something in the field, often you don't need comparisons since there are no existing method to compare. This is not likely unless you perform a ground breaking new innovative research. 


\section{Results}
Results are often summarized using  tables (Table \ref{tab:gm_nodiff}), figures (Figure \ref{fig:meantop}), plots and quantitative numbers. Simply displaying pretty figures is not results. Must have quantitative numbers to demonstrate your method. Also you need to provide an interpretation of the results. Must also provide interpretation of the results. 


\begin{table}[t] 
\caption{Sample table summarizing the performance of the proposed topological loss $\mathcal{L}_{top}$ over other baseline methods. Generated by a former student who took the class \citep{song.2020.arXiv}.}
\label{tab:gm_nodiff}
\centering
\resizebox{\textwidth}{!}{
\begin{tabular}{ccc|ccccc}
    $d$ & $c$ & $p$ & GA & SM & RRWM & IPFP & $\mathcal{L}_{top}$ \\
    \hline 
    \multirow{6}{*}{12 vs. 12} & \multirow{2}{*}{2 vs. 2}   & 0.6 & $ 0.49 \pm 0.27 $ & $ 0.46 \pm 0.30 $ & $ 0.51 \pm 0.30 $ & $ 0.47 \pm 0.28 $ & $ 0.53 \pm 0.29 $ \\
                                 &                            & 0.8 & $ 0.45 \pm 0.25 $ & $ 0.47 \pm 0.31 $ & $ 0.56 \pm 0.29 $ & $ 0.47 \pm 0.30 $ & $ 0.50 \pm 0.30 $ \\
                                 & \multirow{2}{*}{3 vs. 3}   & 0.6 & $ 0.45 \pm 0.32 $ & $ 0.44 \pm 0.26 $ & $ 0.47 \pm 0.27 $ & $ 0.51 \pm 0.30 $ & $ 0.46 \pm 0.31 $ \\
                                 &                            & 0.8 & $ 0.54 \pm 0.31 $ & $ 0.51 \pm 0.27 $ & $ 0.51 \pm 0.29 $ & $ 0.52 \pm 0.29 $ & $ 0.51 \pm 0.30 $ \\
                                 & \multirow{2}{*}{6 vs. 6} & 0.6 & $ 0.57 \pm 0.30 $ & $ 0.51 \pm 0.28 $ & $ 0.56 \pm 0.29 $ & $ 0.45 \pm 0.26 $ & $ 0.58 \pm 0.29 $ \\
                                 &                            & 0.8 & $ 0.55 \pm 0.29 $ & $ 0.48 \pm 0.26 $ & $ 0.52 \pm 0.27 $ & $ 0.54 \pm 0.30 $ & $ 0.49 \pm 0.27 $ \\

    \hline \hline
    
    \multirow{6}{*}{18 vs. 18} & \multirow{2}{*}{2 vs. 2}   & 0.6 & $ 0.48 \pm 0.26 $ & $ 0.49 \pm 0.32 $ & $ 0.54 \pm 0.29 $ & $ 0.47 \pm 0.30 $ & $ 0.54 \pm 0.31 $ \\
                                 &                            & 0.8 & $ 0.52 \pm 0.28 $ & $ 0.50 \pm 0.28 $ & $ 0.46 \pm 0.30 $ & $ 0.52 \pm 0.25 $ & $ 0.50 \pm 0.26 $ \\
                                 & \multirow{2}{*}{3 vs. 3}   & 0.6 & $ 0.49 \pm 0.28 $ & $ 0.58 \pm 0.31 $ & $ 0.43 \pm 0.28 $ & $ 0.51 \pm 0.27 $ & $ 0.53 \pm 0.30 $ \\
                                 &                            & 0.8 & $ 0.46 \pm 0.30 $ & $ 0.51 \pm 0.27 $ & $ 0.52 \pm 0.33 $ & $ 0.45 \pm 0.29 $ & $ 0.53 \pm 0.27 $ \\
                                 & \multirow{2}{*}{6 vs. 6} & 0.6 & $ 0.53 \pm 0.28 $ & $ 0.48 \pm 0.30 $ & $ 0.51 \pm 0.30 $ & $ 0.45 \pm 0.29 $ & $ 0.44 \pm 0.33 $ \\
                                 &                            & 0.8 & $ 0.54 \pm 0.27 $ & $ 0.52 \pm 0.30 $ & $ 0.48 \pm 0.26 $ & $ 0.52 \pm 0.31 $ & $ 0.43 \pm 0.30 $ \\

    \hline \hline
    
    \multirow{6}{*}{24 vs. 24} & \multirow{2}{*}{2 vs. 2}   & 0.6 & $ 0.52 \pm 0.28 $ & $ 0.49 \pm 0.30 $ & $ 0.50 \pm 0.30 $ & $ 0.48 \pm 0.28 $ & $ 0.55 \pm 0.26 $ \\
                                 &                            & 0.8 & $ 0.53 \pm 0.27 $ & $ 0.56 \pm 0.30 $ & $ 0.51 \pm 0.30 $ & $ 0.56 \pm 0.32 $ & $ 0.52 \pm 0.30 $ \\
                                 & \multirow{2}{*}{3 vs. 3}   & 0.6 & $ 0.48 \pm 0.29 $ & $ 0.54 \pm 0.27 $ & $ 0.49 \pm 0.26 $ & $ 0.49 \pm 0.30 $ & $ 0.52 \pm 0.30 $ \\
                                 &                            & 0.8 & $ 0.55 \pm 0.29 $ & $ 0.49 \pm 0.27 $ & $ 0.52 \pm 0.28 $ & $ 0.49 \pm 0.30 $ & $ 0.47 \pm 0.26 $ \\
                                 & \multirow{2}{*}{6 vs. 6} & 0.6 & $ 0.47 \pm 0.30 $ & $ 0.45 \pm 0.31 $ & $ 0.51 \pm 0.29 $ & $ 0.56 \pm 0.28 $ & $ 0.49 \pm 0.29 $ \\
                                 &                            & 0.8 & $ 0.51 \pm 0.30 $ & $ 0.47 \pm 0.28 $ & $ 0.54 \pm 0.28 $ & $ 0.56 \pm 0.31 $ & $ 0.51 \pm 0.31 $ \\

    \hline
    
\end{tabular}}
\end{table}


\begin{figure}
\includegraphics[width=1\linewidth]{toy_meantop.png}
\centering
\caption{\small Figure illustrating how to average graphs of different sizes and topology using persistent homology. Generated by a former student who took the class \citep{song.2023}.}
\label{fig:meantop}
%\vspace{-0.5cm}
\end{figure}







\section{Discussion}
Discuss the limitation of your method. Discuss what you could have done better for future studies. Discuss what you did not do that you could have done. Often putting incomplete results into discussion is a good idea. You can also provide   an alternate strategy in case the proposed method did not perform well.


\section{Conclusion}
Concluding remarks. State what conclusion you obtained from the research. 


\section*{Academic Integrity}
This course follows the University of Wisconsin--Madison policy on academic integrity. Students are responsible for reviewing and complying with Chapter UWS-14 at
\url{https://docs.legis.wisconsin.gov/code/admin_code/uws/14}.


\section{References}
Provide minimum 10-20 relevant references to your method. The journal paper is cited as \citet{chung.2001.NI} for multiple authors or 
\citet{song.2023} for two author papers. Conference paper is cited similarly as \citet{chung.2003.CVPR}. Books are cited as \citet{chung.2012.CNA}. See reference below to see how these citations are listed. You should reference more than 10 papers for research paper. 

\bibliographystyle{plainnat}
\bibliography{reference.2026.01.06}





\end{document}
